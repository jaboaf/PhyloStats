% Template for the submission to: The Annals of Applied Statistics [AOAS]
% In this template, the places where you need to fill in your information are indicated by '???'.
%%%%%%%%%%%%%%%%%%%%%%%%%%%%%%%%%%%%%%%%%%%%%%
\documentclass[aoas]{imsart}
%% Packages
\RequirePackage{amsthm,amsmath,amsfonts,amssymb}
\RequirePackage[authoryear]{natbib}
%\RequirePackage[colorlinks,citecolor=blue,urlcolor=blue]{hyperref}
\RequirePackage{graphicx} % uncomment this for including figures

\startlocaldefs
%%%%%%%%%%%%%%%%%%%%%%%%%%%%%%%%%%%%%%%%%%%%%%
% Uncomment next line to change the type of equation numbering.
%\numberwithin{equation}{section}
%%%%%%%%%%%%%%%%%%%%%%%%%%%%%%%%%%%%%%%%%%%%%%
%% For Axiom, Claim, Corollary, Hypothezis,Lemma, Theorem, Proposition, use \theoremstyle{plain}
%\theoremstyle{plain}
%\newtheorem{???}{???}
%\newtheorem*{???}{???}
%\newtheorem{???}{???}[???]
%\newtheorem{???}[???]{???}
%%%%%%%%%%%%%%%%%%%%%%%%%%%%%%%%%%%%%%%%%%%%%%
%% For Assumption, Definition, Example,Notation, Property, Remark, Fact use \theoremstyle{remark}
%\theoremstyle{remark}
%\newtheorem{???}{???}
%\newtheorem*{???}{???}
%\newtheorem{???}{???}[???]
%\newtheorem{???}[???]{???}
%%%%%%%%%%%%%%%%%%%%%%%%%%%%%%%%%%%%%%%%%%%%%%
%% Please put your definitions here:        %%
%%%%%%%%%%%%%%%%%%%%%%%%%%%%%%%%%%%%%%%%%%%%%%
\endlocaldefs
\begin{document}

\begin{frontmatter}
%% Title of Article %%
\title{Covid Example}
\begin{aug}
%% Authors %%
\author[A]{\fnms{Joseph} \snm{Whaley Aboaf}\ead[label=e1]{jww262@cornell.edu}},
\author[B]{\fnms{Martin} \snm{Wells}\ead[label=e2]{mtw1@cornell.edu}}
%% Addresses %%
\address[A]{Department of Mathematics, Cornell University, \printead{e1}}
\address[B]{Department of Statistics, Cornell University, \printead{e2}}
\end{aug}
\begin{abstract}\end{abstract}
\begin{keyword}\kwd{Covid}\kwd{Data} \end{keyword}
\end{frontmatter}

%%%%%%%%%%%%%%%%%%%%%%%%%%%%%%%%%%%%%%%%%%%%%%
%% Table of Contents %%
\tableofcontents
%% Main text entry area %%
\section{Basic Info}
Birthdate of this document: 06/21/21
Goal:
\item Familiarize myself with biological data, and lots of it.
\item Using latex in a stream of conscious way, i.e. while I am analysing data.
\item try out some methods; at least one thing that comes from inside my head, and at least one thing someone else did.
\item Identify questions i didn't know i had.
\item Figure out what I don't know
Data: Complete Nucleotide Sequences retrieved from the National Center for Biotechnology Information SARS-CoV-2 Data Hub.
They say to cite them with the information below; I have not read this paper.
Virus Variation Resource - improved response to emergent viral outbreaks. Hatcher EL, Zhdanov SA, Bao Y, Blinkova O, Nawrocki EP, Ostapchuck Y, Schaffer AA, Brister JR. Nucleic Acids Res. 2017 Jan 4;45(D1):D482-D490. doi: 10.1093/nar/gkw1065. Epub 2016 Nov 28.
Specifically this is the link % https://www.ncbi.nlm.nih.gov/labs/virus/vssi/#/virus?SeqType_s=Nucleotide&VirusLineage_ss=SARS-CoV-2,%20taxid:2697049&Completeness_s=complete %
Descriptions of FASTA definitions may be found here: % https://blast.ncbi.nlm.nih.gov/Blast.cgi?CMD=Web&PAGE_TYPE=BlastDocs&DOC_TYPE=BlastHelp %

\section{Data}
I chose to only include, "complete" sequences. There are 356,471 of these. Let M = 356,471
There are some variables available:
\item Accesion: This seems to uniquely identify a sequence
\item Submitters: This is a list of names of the people attached to the submission of the sequence.
\item Release Date: This is the date the submission was released I think.
\item Pangolin: unsure.
\item Species: This takes the value "Severe acute respiratory syndrome-related coronavirus"
\item Molecule Type: This takes the values "ssRNA(+)"
\item Length: Some Integer. I am wondering why these are not all multiples of 3.
\item Geo Location: Of the form "Country:City"
\item USA: If given, this is the state of the sample.
\item Host: This takes the values "Homo sapiens", "Felis Catus", and "Canis Lupus Familiaris"
\item Isolation Source: Where/how the sample was collected from the source, e.g. swab.
\item Collection Date: Date of collection, in any of the following formats, YYYY, YYYY-MM, or YYYY-MM-DD
\item Sequence: This is some word on the alphabet: 'A','C','G','T','N'. 'N' denotes "unknown nucleic acid residue". There seem to be quite a bit of 'N's. Approximately, 10\% of the first sequence is 'N'.

I am interested in a subset of the variables above.
\item Location: This is a new variable which 
\item Collection Date

Question(s) for data:
\item What is the distribution of 'N's throughout the sample?
\item Is each sequence unique?
\item Is each observation unique?

\section{Statistical Interpretations}
The objects of analysis are sequences $\{w_m\}_{m=1}^M = W$, i.e. each $w_m$ corresponds to an observation and is a word over the alphabet $L=\{A,C,G,T\}$. The set of all words on the letters $L$ is denoted $L^*$.

Our data is contained in:
$V_L^{\otimes K} $ where $K := \max_w \#w $ is the maximum length of a sequence, and $V_L := sp\{e_A, e_C, e_G,e_T\} $ 
$S:L^* \rightarrow L^* $ given by $Sw = Res^{S_\infty}_{S_{\#w} }w$
$S_L $ is the symmetric group on L. $g \in S_L$ may act on a word $w$, from the right by replacing each instance of l by $g(l)$ 


Questions:
\item @Marty, how do you feel about the use of $\#$ for the "length" function ? I think it might be good if we built up some consistent notation in consideration of the fact that using A, C, G, or T (and N) for something other than reference to the letters might  be confusing. I propose:
A for 'A' in nucleotide sequences
C for 'C' in nucleotide sequences
G for 'G' in nucleotide sequences
L for a set of letters, l for a letter.
T for 'T' in nucleotide sequences
V for vector spaces,
W for a set of words, w for a word.




%%%%%%%%%%%%%%%%%%%%%%%%%%%%%%%%%%%%%%%%%%%%%%
%% Support information, if any, should be provided in the Acknowledgements section.
%% \begin{acks}[Acknowledgments] The authors would like to thank ... \end{acks}
%%%%%%%%%%%%%%%%%%%%%%%%%%%%%%%%%%%%%%%%%%%%%%
%% Funding information %%
%% \begin{funding} NSF \end{funding}

\begin{thebibliography}{}
% \bibitem[\protect\citeauthoryear{???}{???}]{b1}
\end{thebibliography}

\end{document}